\documentclass[11pt, oneside]{article}   	% use "amsart" instead of "article" for AMSLaTeX format
\usepackage{geometry}                		% See geometry.pdf to learn the layout options. There are lots.
\geometry{letterpaper}                   		% ... or a4paper or a5paper or ... 
\usepackage{color}
\usepackage[parfill]{parskip}    		% Activate to begin paragraphs with an empty line rather than an indent
\usepackage{graphicx}				% Use pdf, png, jpg, or eps§ with pdflatex; use eps in DVI mode
								% TeX will automatically convert eps --> pdf in pdflatex		
\usepackage{amssymb}



\title{Response to reviews of ``Extreme heterogeneity of influenza virus infection in single cells'' for \textit{eLife}}
\author{Alistair B. Russell, Cole Trapnell, and Jesse D. Bloom}

\begin{document}
\maketitle

\emph{Below, the original comments {\color{blue} are in blue}, and our responses are in black.}

\color{blue}

\section*{SUMMARY} 

Your manuscript describes the use of a novel single cell RNAseq method to examine heterogeneity in viral mRNA production within single cells infected with single influenza A virions. You report that viral mRNA levels, as a proportion of total cellular mRNA, can vary significantly between infected cells, and this variability cannot be simply explained by the activity of innate anti-viral defenses. Further, correlating the levels of specific host transcripts with viral transcript levels allows you to identify pathways that tend to correlate with viral levels. 

The strengths of the paper include the description of a novel method for leveraging single cell cRNAseq to quantify heterogeneity and stochasticity in viral gene expression during infection. This is a really important question for the field, and the approaches detailed here will likely by adopted by other groups studying other viruses. The authors do a good job of accounting for many of the pitfalls inherent in the experimental execution and data analysis. The manuscript is well written, and does an especially effective job of visually presenting complex datasets. 

The main shortcoming of the paper is that it provides very little in the way of new information. As the authors clearly and honestly point out, most of the findings in this paper are simply confirming observations made in older papers. Additionally, the findings are purely descriptive, and provide little insight into the mechanisms that may give rise to the observations. 

We believe that addressing the points below would help ameliorate some of the shortcomings of the paper. 

\section*{MAJOR COMMENTS} 

1. There are some obvious sources of variability that haven't been suitably discussed. 

- The first is variation in the timing of the early stages of infection. No steps appear to have been taken to synchronize infection, or to limit secondary spread of the virus within the cultures. Could heterogeneity in the timing of binding/entry/fusion/trafficking explain a lot of the variation observed? 

- Another potential contributor is the cell cycle status of the infected cells. Did the host transcript data shed light on whether cell cycle status influenced viral transcription levels? This is especially important to address given that some of the genes showing association with the viral burden are cell-cycle related. 

- In assessing to what extent lack of RNP expression accounts for the viral mRNA expression variability (page 6, paragraph 2), we think it is important to take into account potential extracellular contamination. Specifically, it would be more convincing if the analysis omitted cells with mixed wildtype/synonymous clones. 

- As the variability of the viral RNA load appears to be the central result of the manuscript, it would be useful to: 

    a. Employ a statistic to quantify the variability (e.g. entropy or Gini index) 

    b. Use simple models to illustrate how much variability can be accounted for by simple effects, such as expected Poisson co-infection frequency, or the likelihood of attaining full complement of RNP genes. 

2. While the silent tagging method used to address co-infection is clever and appreciated, the issue is not fully settled. The dismissal of co-infection as a factor influencing cell-to-cell variation in lines 127-9 is based on too few cells to draw any conclusions, and thus needs to be tempered. Also, there is likely to be a significant amount of cryptic co-infection with identical barcode viruses (expected to be similar to that of mixed barcodes, ~10\%) that could influence measured heterogeneity. These points should be made in the text. 

3. The analysis of host cell transcripts positively or negatively associated with high viral mRNA expression is pretty minimal. Do the host genes identified here match up with the results of previous studies that screened for host pro- and anti-viral factors (reviewed in Watanabe et al. DOI: 10.1016/j.chom.2010.05.008)? Also, targeted gene knockdown or over-expression experiments could help establish the causality underlying these relationships. 

4. All analysis of viral gene expression seems to be at the segment level. How do different transcripts expressed from the same gene segment compare (i.e. NS1 and NEP)? 

5. The determination of the minimum required influenza fraction (Figure 4) is based on sound logic; however, we are concerned that the observed results do not fully align with this model. Specifically, while the 10hr experiment looks reasonable, the distributions of wildtype/synonymous mixed cells in other experiments do not appear to show the same trend (most notable for 6hr and 8hr samples, where almost no mixed cells are observed at low fractions). In that regard, using the 10hr dataset to estimate thresholds for all of the other samples does not seem appropriate. 

\section*{MINOR COMMENTS}

- Fig 2: Normalizing high versus low DI stocks based on equal TCID50 to compare IFN induction is not appropriate since these stocks will differ in total particle number. In this experiment, the cells would receive a much higher overall dose of viral RNAs with the high DI stock compared to the others. 

- Fig. 3B: While I appreciate the density of information presented by this figure, the authors' main point concerning heterogeneity in the viral proportion of total mRNA reads might be made more clearly by simply plotting the relative proportion of mRNAs that are viral per cell. 

- Line 110: Should say "likely reflecting co-infection" 

- Lines 186-8: This statement doesn't make much sense since you are talking about variation in mRNA proportion. Global effects such as cell size should not affect proportion. 

- Lines 193-8: Is this really remarkable? As the authors point out, there is an extensive body of work demonstrating complementation during IAV infection. 

- Line 329: "characterizing" misspelled. 

- Lines 330-4: This statement about a confluence of factors leading to extreme cell-to-cell heterogeneity should be reworded since the data here actually doesn't show a clear role for innate immunity in promoting heterogeneity and there is no significant evaluation of the effect of host cell state on heterogeneity. 

- The "replicate" 8-hr sample isn't really a replicate since a different MOI was used. Also, the purpose of the differences in treatment for this sample compared with the 8-hr sample are not well explained. 

-Figure 3 is not effective in supporting the observations made in the text (that most cells have little viral RNA, and that there's a wide variability in the amount of viral RNA). Separate distributions for the viral RNA would be more telling. Log scale would be more appropriate for showing counts per cell. 

-Figure 4C: the threshold line appears to be a bit to the right of the global maximum - going through the description, it was not clear why that should be the case. 

-Why was media change skipped in the replicate experiment? 

-Consider using different color schemes for Fig 4A and B, as different variables are being shown. 

-Figure 5A legend: axis descriptions appear to be swapped




\end{document}  