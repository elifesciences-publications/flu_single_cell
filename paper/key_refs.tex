%%%%%%%%%%%%%%%%%%%%%%%%%%%%%%%%%%%%%%%%%%%%%%%%%%%%%%%%%%%%
%%% ELIFE ARTICLE TEMPLATE
%%%%%%%%%%%%%%%%%%%%%%%%%%%%%%%%%%%%%%%%%%%%%%%%%%%%%%%%%%%%
%%% PREAMBLE 
\documentclass[9pt,lineno]{elife}
% Use the onehalfspacing option for 1.5 line spacing
% Use the doublespacing option for 2.0 line spacing
% Please note that these options may affect formatting.

\usepackage{lipsum} % Required to insert dummy text
\usepackage[version=4]{mhchem}
\usepackage{siunitx}
\DeclareSIUnit\Molar{M}

\begin{document}



\section{Key background references}
Here we list some key background references on some relevant points.

\subsection{Stochasticity during viral infection}
\begin{itemize}

\item \citet{delbruck1945burst} showed that the amount of progeny virions released from bacteria infected with bacteriophage differed by over 2 orders of magnitude.

\item \citet{zhu2009growth} showed that the number of progeny virions produced by mammalian cells infected with VSV differ by over 2 orders of magnitude. 
They further suggest that the differences are mostly due to cellular factors, since they are not heritable.

\item \citet{schulte2014single} examine single cells infected with poliovirus.
They found that the amount of both positive-sense and negative-sense genome and progeny virus differed by over an order of magnitude across cells.
Higher MOI was associated with more positive-sense genome but \emph{not} higher yields of progeny virus.

\item \citet{akpinar2016high} examine VSV infection of single cells both in the presence and absence of defective interfering particles (DIPs). 
They track both the amount of produced viral progeny and fluorescence of a reporter.
One interesting finding is that the viral yield from individual isolated cells is consistently lower than that from populations of cells, suggesting that presence of some cell-cell signaling or secreted factor might help viral production.
They find that adding more DIPs decreases both viral production and the expression of the reporter gene, although the former is decreased more.

\item \citet{combe2015single} also look at the production of VSV from single infected cells, although immediately some of these cells are likely multiply infected. 
They find that the yield of viral progeny differs by about two orders of magnitude across cells.

\item \citet{heldt2015single} examine stochasticity during influenza virus infection.
They show that the amount of infectious progeny virions produced per infection (at high MOI) ranged from one to almost 1000.
They also show that the amount of vRNA per cell (again at high MOI infection) ranges by almost 3 orders of magnitude.
The amount of vRNA and progeny virions is strongly correlated.
For most segments, the amounts of different mRNAs is correlated.
They also have a bunch of simulations that I looked at less closely.

\item \citet{dou2017analysis} use an RNA-labeling method to look at the presence of vRNAs in single infected cells.
By their method, they find that only about 20\% of cells contain all 8 vRNPs.
They also show that there is a very limited window for co-infection.

\end{itemize}

\subsection{Influenza co-infection and defective particles}
\begin{itemize}

\item \citet{davis1980influenza} is an old paper that appears to be the original one showing that influenza defective RNAs have both termini and so represent some type of internal deletion.

\item \citet{odagiri1997segment} look at a PA defective segment in the presence of an NS2 mutant.
They show that the shorter defective PA segment is favored at both the levels of segment replication and genome packaging. 
I remain a bit unclear about how the NS2 mutant plays into all of this, other than that apparently this NS2 mutant generates lots of PA defectives...

\item \citet{hutchinson2008mutational} look at how many infected cells lack expression of at least one of two segments tested.
They find that for wildtype virus this is about 20\%, but when there are synonymous mutations in the packaging signal of M, then it is about 50\%.

\item \citet{saira2013sequence} shows that defective influenza particles are present in infected humans.

\item \citet{brooke2013most} show that over half of cells infected with PR8 fail to express one of four proteins tested.
High MOI infection alleviates this incomplete expression, presumably via co-infection with viruses each of which fails to express different genes.

\item \citet{brooke2014influenza} find a mutation in NP that specifically reduces the packaging of NA in a way that is advantageous in guinea pigs.
They do various things showing that co-infection occurs \textit{in vivo}, and also show that there is at least some amount of dose-response in terms of gene expression as cells that get lots of no-NA virus during co-infection express relatively less NA protein.


\end{itemize}

\subsection{Interferon induction}
\begin{itemize}

\item \citet{perez2014unbiased} looks at type I IFN induction in A549 cells with a fluorescent reporter.
The wildtype virus induces IFN in a few percent of cells.
A NS1-deletion virus induces in 15-20\% of cells.
They passage in IFN-deficient cells to isolate various mutants that induce more than wildtype but less than the delta-NS1 virus.
Most of these mutations are in proteins other than NS1, including some in M1 that were shown to induce IFN more effectively.
They also argue that some of the inducing mutations lead to more defective segments to accumulate.

\item \citet{killip2015influenza} reviews IFN induction by influenza, including the point that only a fraction of cells induce IFN upon infection.

\item \citet{killip2017single} look at IFN induction by wildtype and NS1-defective influenza virus.
Although some viruses (like Sendai) are able to induce IFN expression in almost all their cells, influenza only induces IFN in less than one percent of cells.
An NS1-defective virus induced IFN in many more cells, but it was still only 5 to 20\%.
In neither case is there an obvious correlation between NP protein production and IFN activation -- IFN is both present and absent in cells with little and lots of NP.

\item \citet{shalek2013single} and \citet{shalek2014single} using single-cell RNA-seq to show that interferon induction is highly variable across individual cells in bone-marrow derived dendritic cells.
Some cells express lots of IFN related genes, others express almost none.
They suggest that some ``precocious'' cells are responsible for much of the IFN signaling.

\item \citet{bhushal2017cell} look at expression of IFN in single cell using an RFP reporter.
In epithelial cells, type I IFN induction is bimodal, with both the fraction of cells responding and the magnitude of response increasing with more IFNbeta, and saturating at near 100\% response.
In contrast, for type III IFN, although the response is also bimodal, it saturates at less than 100\%, and there is less dose-response in the magnitude.
They also did interesting experiments looking at ``memory'' in the response by separating responders and non-responders and then re-stimulating.
Cells that did not express IFN first time still had a fraction the responded the second time, whereas responders the first time usually although not always responded the second time.
So there is some but incomplete memory among responders.
They show that at least for type III IFN, the difference between responders and non-responders is not at the level of STAT1 activation, but is rather downstream and may involve histone acetylation as it is affected by HDAC inhibitors.
They also look in organoids and polarized cells, and find that in this more physiological setting the cells are more type III IFN responsive.

\item \citet{lopez2014defective} reviews the idea that defective genomes are important for IFN induction by many viruses including influenza.

\item \citet{baum2010preference} argues that RIG-I preferentially binds to short (defective) influenza segments.

\item \citet{tapia2013defective} shows that mice infected with influenza with more defective particles have more IFN induction.

\end{itemize}

\subsection{Transcriptional dynamics}

\begin{itemize}

\item \citet{hatada1989control} examine the accumulation of each influenza mRNA by Northern blot.
They find the same general hierarchy of expression as us: M and NS are the highest, then NP, then NA, then HA, then the three polymerases.
Specifically estimates that the polymerase proteins reach about $10^3$ copies / cell, HA gets to about $7 \times 10^3$ per cell, NA to about $10^4$ per cell, NP to about $1.5 \times 10^4$ per cell, NS to almost $2 \times 10^4$ / cell, and M to over $2 \times 10^4$ cell.

\citet{vreede2004model} This paper examines production of mRNA and cRNA in the presence of cycloheximide.
They demonstrate the cRNA production requires the presence of the polymerase proteins (PB2, PB1, PA, NP) in excess of those in the initial vRNAs, although the PB1 can be inactive. 

\item \citet{kawakami2011strand} use qPCR to look at the accumulation of NP and NA in infected cells.
They suggest that NP mRNA reaches a peak around 6 hours of $5 \times 10^4$ / cell, whereas NA reaches a much lower peak and after more time.
The paper also shows that mRNA levels peak at 4-6 hours regardless of the presence or absence of cycloheximide, but the mRNA production is less in the presence of cycloheximide indicating secondary transcription.

\item \citet{shapiro1987influenza} examines gene expression using Northern blots.
They find that the \emph{rate} of mRNA synthesis (at least for NP, M1, and NS1) generally peaks around 2.5 hours, with protein levels appearing to be near maximal by about 5.5 hours.
The rate of mRNA synthesis had fallen to 5\% of the maximal rate by 4.5 hours. 
Note that these times are for infection intiated \emph{after} absorption has been allowed to occur.

\end{itemize}



\nocite{*} % This command displays all refs in the bib file
\bibliography{references}


\end{document}
